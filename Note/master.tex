\documentclass[a4paper]{report}
% basics
\usepackage[utf8]{inputenc}
\usepackage[T1]{fontenc}
\usepackage{textcomp}
\usepackage[hyphens]{url}
\usepackage[colorlinks=true,linkcolor=cyan,urlcolor=magenta,citecolor=violet]{hyperref}
\usepackage{graphicx}
\usepackage{float}
\usepackage{booktabs}
\usepackage{emptypage}
\usepackage{subcaption}
\usepackage[usenames,dvipsnames]{xcolor}
\input{../Math.tex}


\usepackage{amsmath, amsfonts, mathtools, amsthm, amssymb}
\usepackage{geometry}
\usepackage{mathrsfs}
\usepackage{cancel}
\usepackage{systeme}

\usepackage[inline, shortlabels]{enumitem}
\usepackage{multicol}
\setlength\multicolsep{0pt}

\usepackage{caption}
\captionsetup{belowskip=0pt}
\geometry{a4paper,left=2.54cm,right=2.54cm,top=2.54cm,bottom=2.54cm}

% for the big braces
\usepackage{bigdelim}

% correct
\definecolor{correct}{HTML}{009900}
\newcommand\correct[2]{\ensuremath{\:}{\color{red}{#1}}\ensuremath{\to }{\color{correct}{#2}}\ensuremath{\:}}
\newcommand\green[1]{{\color{correct}{#1}}}

% hide parts
\newcommand\hide[1]{}

% si unitx
\usepackage{siunitx}
\sisetup{locale = FR}
% \renewcommand\vec[1]{\mathbf{#1}}
\newcommand\mat[1]{\mathbf{#1}}


% tikz
\usepackage{tikz}
\usetikzlibrary{intersections, angles, quotes, positioning}
\usetikzlibrary{arrows.meta}
\usepackage{pgfplots}
\pgfplotsset{compat=1.13}


\tikzset{
	force/.style={thick, {Circle[length=2pt]}-stealth, shorten <=-1pt}
}



% Clear out the item autoref name
\makeatletter
\def\itemautorefname{\@gobble}
\makeatother

% theorems
\makeatother
\usepackage{thmtools}
\usepackage[framemethod=TikZ]{mdframed}

\mdfsetup{skipabove=1em,skipbelow=0em}

\theoremstyle{definition}

\declaretheoremstyle[
	headfont=\bfseries\sffamily\color{ForestGreen!70!black}, bodyfont=\normalfont,
	mdframed={
			linewidth=2pt,
			rightline=false, topline=false, bottomline=false,
			linecolor=ForestGreen, backgroundcolor=ForestGreen!5,
			nobreak=false
		}
]{thmgreenbox}

\declaretheoremstyle[
	headfont=\bfseries\sffamily\color{ForestGreen!70!black}, bodyfont=\normalfont,
	mdframed={
			linewidth=2pt,
			rightline=false, topline=false, bottomline=false,
			linecolor=ForestGreen, backgroundcolor=ForestGreen!8,
			nobreak=false
		}
]{thmgreen2box}

\declaretheoremstyle[
	headfont=\bfseries\sffamily\color{NavyBlue!70!black}, bodyfont=\normalfont,
	mdframed={
			linewidth=2pt,
			rightline=false, topline=false, bottomline=false,
			linecolor=NavyBlue, backgroundcolor=NavyBlue!5,
			nobreak=false
		}
]{thmbluebox}

\declaretheoremstyle[
	headfont=\bfseries\sffamily\color{TealBlue!70!black}, bodyfont=\normalfont,
	mdframed={
			linewidth=2pt,
			rightline=false, topline=false, bottomline=false,
			linecolor=TealBlue,
			nobreak=false
		}
]{thmblueline}

\declaretheoremstyle[
	headfont=\bfseries\sffamily\color{RawSienna!70!black}, bodyfont=\normalfont,
	mdframed={
			linewidth=2pt,
			rightline=false, topline=false, bottomline=false,
			linecolor=RawSienna, backgroundcolor=RawSienna!5,
			nobreak=false
		}
]{thmredbox}

\declaretheoremstyle[
	headfont=\bfseries\sffamily\color{RawSienna!70!black}, bodyfont=\normalfont,
	mdframed={
			linewidth=2pt,
			rightline=false, topline=false, bottomline=false,
			linecolor=RawSienna, backgroundcolor=RawSienna!8,
			nobreak=false
		}
]{thmred2box}

\declaretheoremstyle[
	headfont=\bfseries\sffamily\color{SeaGreen!70!black}, bodyfont=\normalfont,
	mdframed={
			linewidth=2pt,
			rightline=false, topline=false, bottomline=false,
			linecolor=SeaGreen, backgroundcolor=SeaGreen!2,
			nobreak=false
		}
]{thmgreen3box}

\declaretheoremstyle[
	headfont=\bfseries\sffamily\color{WildStrawberry!70!black}, bodyfont=\normalfont,
	mdframed={
			linewidth=2pt,
			rightline=false, topline=false, bottomline=false,
			linecolor=WildStrawberry, backgroundcolor=WildStrawberry!5,
			nobreak=false
		}
]{thmpinkbox}

\declaretheoremstyle[
	headfont=\bfseries\sffamily\color{MidnightBlue!70!black}, bodyfont=\normalfont,
	mdframed={
			linewidth=2pt,
			rightline=false, topline=false, bottomline=false,
			linecolor=MidnightBlue, backgroundcolor=MidnightBlue!5,
			nobreak=false
		}
]{thmblue2box}

\declaretheoremstyle[
	headfont=\bfseries\sffamily\color{Gray!70!black}, bodyfont=\normalfont,
	mdframed={
			linewidth=2pt,
			rightline=false, topline=false, bottomline=false,
			linecolor=Gray, backgroundcolor=Gray!5,
			nobreak=false
		}
]{notgraybox}

\declaretheoremstyle[
	headfont=\bfseries\sffamily\color{Gray!70!black}, bodyfont=\normalfont,
	mdframed={
			linewidth=2pt,
			rightline=false, topline=false, bottomline=false,
			linecolor=Gray,
			nobreak=false
		}
]{notgrayline}

% \declaretheoremstyle[
% 	headfont=\bfseries\sffamily\color{RawSienna!70!black}, bodyfont=\normalfont,
% 	numbered=no,
% 	mdframed={
% 			linewidth=2pt,
% 			rightline=false, topline=false, bottomline=false,
% 			linecolor=RawSienna, backgroundcolor=RawSienna!1,
% 		},
% 	qed=\qedsymbol
% ]{thmproofbox}

\declaretheoremstyle[
	headfont=\bfseries\sffamily\color{NavyBlue!70!black}, bodyfont=\normalfont,
	numbered=no,
	mdframed={
			linewidth=2pt,
			rightline=false, topline=false, bottomline=false,
			linecolor=NavyBlue, backgroundcolor=NavyBlue!1,
			nobreak=false
		}
]{thmexplanationbox}

\declaretheoremstyle[
	headfont=\bfseries\sffamily\color{WildStrawberry!70!black}, bodyfont=\normalfont,
	numbered=no,
	mdframed={
			linewidth=2pt,
			rightline=false, topline=false, bottomline=false,
			linecolor=WildStrawberry, backgroundcolor=WildStrawberry!1,
			nobreak=false
		}
]{thmanswerbox}

\declaretheoremstyle[
	headfont=\bfseries\sffamily\color{Violet!70!black}, bodyfont=\normalfont,
	mdframed={
			linewidth=2pt,
			rightline=false, topline=false, bottomline=false,
			linecolor=Violet, backgroundcolor=Violet!1,
			nobreak=false
		}
]{conjpurplebox}

\declaretheorem[style=thmgreenbox, name=Definition, numberwithin=section]{definition}
\declaretheorem[style=thmgreen2box, name=Definition, numbered=no]{definition*}
\declaretheorem[style=thmredbox, name=Theorem, numberwithin=section]{theorem}
\declaretheorem[style=thmred2box, name=Theorem, numbered=no]{theorem*}
\declaretheorem[style=thmredbox, name=Lemma, numberwithin=section]{lemma}
\declaretheorem[style=thmredbox, name=Proposition, numberwithin=section]{proposition}
\declaretheorem[style=thmredbox, name=Corollary, numberwithin=section]{corollary}
\declaretheorem[style=thmpinkbox, name=Problem, numberwithin=section]{problem}
\declaretheorem[style=thmpinkbox, name=Problem, numbered=no]{problem*}
\declaretheorem[style=thmblue2box, name=Claim, numbered=no]{claim}
\declaretheorem[style=conjpurplebox, name=Conjecture, numberwithin=section]{conjecture}

% Redefine proof environment to get a full control. 
\makeatletter
\renewenvironment{proof}[1][\proofname]{\par
	\pushQED{\qed}%
	\normalfont \topsep-2\p@\@plus6\p@\relax
	\trivlist
	\item[\hskip\labelsep
	            \color{RawSienna!70!black}\sffamily\bfseries
	            #1\@addpunct{.}]\ignorespaces
	\begin{mdframed}[linewidth=2pt,rightline=false, topline=false, bottomline=false,linecolor=RawSienna, backgroundcolor=RawSienna!1]
		}{%
		\popQED\endtrivlist\@endpefalse
	\end{mdframed}
}
\makeatother

\declaretheorem[style=thmbluebox, numbered=no, name=Example]{eg}
\declaretheorem[style=thmexplanationbox, numbered=no, name=Proof]{tmpexplanation}
\newenvironment{explanation}[1][]{\vspace{-10pt}\pushQED{\(\circledast\)}\begin{tmpexplanation}}{\null\hfill\popQED\end{tmpexplanation}}

\declaretheorem[style=thmblueline, numbered=no, name=Remark]{remark}
\declaretheorem[style=thmblueline, numbered=no, name=Note]{note}
\declaretheorem[style=thmpinkbox, numbered=no, name=Exercise]{exercise}
\declaretheorem[style=notgrayline, numbered=no, name=As previously seen]{prev}
\declaretheorem[style=thmgreen3box, numbered=no, name=Intuition]{intuition}
\declaretheorem[style=notgraybox, numbered=no, name=Notation]{notation}
\declaretheorem[style=thmanswerbox, numbered=no, name=Answer]{tmpanswer}
\newenvironment{answer}[1][]{\vspace{-10pt}\pushQED{\(\circledast\)}\begin{tmpanswer}}{\null\hfill\popQED\end{tmpanswer}}


\usepackage{etoolbox}
\renewcommand{\qed}{\null\hfill\(\blacksquare\)}

\makeatletter

\def\testdateparts#1{\dateparts#1\relax}
\def\dateparts#1 #2 #3 #4 #5\relax{
	\marginpar{\small\textsf{\mbox{#1 #2 #3 #5}}}
}

\def\@lecture{}%
\newcommand{\lecture}[3]{
	\ifthenelse{\isempty{#3}}{%
		\def\@lecture{Lecture #1}%
	}{%
		\def\@lecture{Lecture #1: #3}%
	}%
	\section*{\@lecture}
	\marginpar{\small\textsf{\mbox{#2}}}
}
\usepackage{pgffor}%
\newcommand{\lec}[2]{%
	\foreach \c in {#1,...,#2}{%
			\IfFileExists{Lectures/lec_\c.tex} {%
				\input{Lectures/lec_\c.tex}%
			}{}%
		}%
}

% fancy headers
\usepackage{fancyhdr}
\pagestyle{fancy}

% LE: left even
% RO: right odd
% CE, CO: center even, center odd
\fancyhead[LE,RO]{Pingbang Hu}

\fancyhead[RO,LE]{\@lecture} % Right odd,  Left even
\fancyhead[RE,LO]{}          % Right even, Left odd
\fancyfoot[RO,LE]{\thepage}  % Right odd,  Left even
\fancyfoot[RE,LO]{}          % Right even, Left odd
\fancyfoot[C]{\leftmark}     % Center

\makeatother

% notes
\usepackage[color=pink]{todonotes}
\usepackage{marginnote}
\let\marginpar\marginnote

% Fix some stuff
% %http://tex.stackexchange.com/questions/76273/multiple-pdfs-with-page-group-included-in-a-single-page-warning
\pdfsuppresswarningpagegroup=1

% Appendix environment
\usepackage{appendix}
\def\chapterautorefname{Section}
\def\sectionautorefname{Section}
\def\appendixautorefname{Appendix}
\renewcommand\appendixname{Appendix}
\renewcommand\appendixtocname{Appendix}
\renewcommand\appendixpagename{Appendix}
% begin appendix autoref patch [\autoref subsections in appendix](https://tex.stackexchange.com/questions/149807/autoref-subsections-in-appendix)
\makeatletter
\patchcmd{\hyper@makecurrent}{%
	\ifx\Hy@param\Hy@chapterstring
		\let\Hy@param\Hy@chapapp
	\fi
}{%
	\iftoggle{inappendix}{%true-branch
		% list the names of all sectioning counters here
		\@checkappendixparam{chapter}%
		\@checkappendixparam{section}%
		\@checkappendixparam{subsection}%
		\@checkappendixparam{subsubsection}%
		\@checkappendixparam{paragraph}%
		\@checkappendixparam{subparagraph}%
	}{}%
}{}{\errmessage{failed to patch}}

\newcommand*{\@checkappendixparam}[1]{%
	\def\@checkappendixparamtmp{#1}%
	\ifx\Hy@param\@checkappendixparamtmp
		\let\Hy@param\Hy@appendixstring
	\fi
}
\makeatletter

\newtoggle{inappendix}
\togglefalse{inappendix}

\apptocmd{\appendix}{\toggletrue{inappendix}}{}{\errmessage{failed to patch}}
\apptocmd{\subappendices}{\toggletrue{inappendix}}{}{\errmessage{failed to patch}}
% end appendix autoref patch

\setcounter{tocdepth}{1}

\author{Sami Yousef}
\title{CO 250 Stuff I Need to Memorize}

\thispagestyle{empty}

\usepackage{adjustbox}
\usepackage[bb=boondox]{mathalfa}
\usepackage{centernot}
\usepackage{algorithm}
\usepackage[noend]{algpseudocode}
\newcommand{\m}[1]{\begin{pmatrix}
	#1
\end{pmatrix}}
\DeclarePairedDelimiter{\floor}{\lfloor}{\rfloor}

\begin{document}
	\maketitle

	\tableofcontents

	\chapter{Linear Programs}
	\section{Certificates}
	For any LP, we know it must have one of three outcomes:
	\begin{enumerate}
		\item \textbf{Optimal}: The LP is feasible and the optimal value is attained.

		\item \textbf{Infeasible}: The LP is infeasible.

		\item \textbf{Unbounded}: The LP is feasible but unbounded.
	\end{enumerate}

	We can find a certificate for each of these cases.

	\begin{definition*}
		\textbf{Certificate of Optimality} \item
		\begin{definition}[Certificate of Optimality I (Bound)]
			\label{optimality:i} If the objective function is of the form $\underbrace{c^T}
			_{<0}\underbrace{x}_{\geq 0}+ z$, then we can say $z$ is an upperbound (or
			lowerbound if LP is min) on the objective function and if we have a
			solution with value $z$, then it must be optimal.
		\end{definition}

		\item
		\begin{definition}[Certificate of Optimality II (Weak Duality)]
			\label{optimality:ii} Let $x$ and $y$ be solutions to the primal (P) and dual
			(D) LPs, respectively. Then we have $x$ and $y$ are optimal for (P) and (D)
			$\Longleftrightarrow c^{T}x + z_{1} = b^{T}y + z_{2}$, where
			$c^{T}x + z_{1}$ is the objective function of (P) and $b^{T}y + z_{2}$ is
			the objective function of (D).
		\end{definition}

		\item
		\begin{definition}[Certificate of Optimality III (CS Conditions)]
			\label{optimality:iii} Let $x$ and $y$ be solutions to the primal (P) and dual
			(D) LPs, respectively. Then we have $x$ and $y$ are optimal for (P) and (D)
			$\Longleftrightarrow$ $x$ and $y$ are feasible and satisfy all the CS conditions.
		\end{definition}
	\end{definition*}

	\begin{definition}[Certificate of Infeasibility (Farkas' Lemma)]
		\label{infeasibility} Let $A$ be the constraint matrix of an LP in SEF. Then
		we have the LP is infeasible $\Longleftrightarrow$ there exists a vector $y$
		such that $y^{T} A \geq \mathbb{0}^{T}$ and $y^{T} b < 0$. This $y$ vector can
		be found at the end of the Simplex Algorithm: If the ending basis is $B$, then
		$y^{T} = c_{B}^{T} A_{B}^{-1}$. If the dual has a feasible solution, then that
		solution can also be used as a certificate.
	\end{definition}

	\begin{definition}[Certificate of Unboundedness]
		\label{unboundedness} A max LP is unbounded if and only if there exists a family
		of solutions $x(t) = \bar x + td$ for some feasible solution $\bar x$ and vector
		$d$ such that:
		\begin{enumerate}
			\item $d \geq 0$

			\item $Ad = \mathbb{0}$.

			\item $c^{T} d > 0$.
		\end{enumerate}

		More generally, the LP is unbounded if and only if there exists a family of
		solutions $x(t)$ such that $x(t)$ is feasible for all $t \geq 0$ and
		$c^{T}x(t) \rightarrow \infty$ (or $-\infty$ if min) as
		$t \rightarrow \infty$.

		$\bar x$ and $d$ can be found during the Simplex Algorithm when the values
		in the pivot column are all non-positive. $\bar x$ is the solution to the LP
		before entering the variable and $d$ is the pivot column.
	\end{definition}

	\section{The Simplex Algorithm}
	For this section, we will assume the LP is in SEF:
	$\{\max c^{T} x : Ax = b, x \geq 0\}$. If we are given a starting basis or BFS,
	we can skip Phase I and proceed with Phase II.

	\begin{definition}[Simplex: Phase I]
		\label{phase:i} In Phase I, we add auxiliary variables and construct an
		auxiliary LP with the following form:
		\begin{equation*}
			\begin{aligned}
				\min              & (\mathbb{0}, \mathbb{1}) \m{x \\ a} \\
				\text{subject to} & \m{A & I} \m{x \\ a} = b            \\
				                  & \m{x \\ a} \geq 0
			\end{aligned}
		\end{equation*}
		where $I$ is the identity matrix and $a$ are the auxiliary variables. Then
		we solve the auxiliary problem using the auxiliary variables as the starting
		basis. We can only proceed to Phase II if the optimal value at the end is zero.
		Otherwise, the original LP is infeasible.
		\begin{note}
			If the original LP contains a constraint $i$ where $b_{i} < 0$, then we
			must multiply that whole constraint by $-1$ before constructing the
			auxiliary LP.
		\end{note}
		\begin{tip}
			During the first iteration of Phase I, we can quickly get the auxiliary LP
			into canonical form by subtracting all the constraints from the objective
			function. Can you see why?
		\end{tip}
	\end{definition}

	\begin{definition}[Simplex: Phase II]
		In Phase II, we solve the original LP using the auxiliary variables as the
		starting basis. In each iteration, we enter the index left-most positive
		value of $c^{T}$ into the basis, and depart the index of $\min \frac{b_{i}}{A_{ji}}$
		where $j$ is the index of the entering variable. After entering and
		departing a variable, we convert the LP into canonical form and repeat until
		we're done. The canonical form can be achieved by:
		\begin{enumerate}
			\item Finding $y^{T} = c_{B}^{T} A_{B}^{-1}$.

			\item Replacing the objective function, $c^{T} x + z_{1}$, with $[c^{T} - y
				^{T} A]x + z_{1} + y^{T} b$.

			\item Multiplying the constraints by $A_{B}^{-1}$.
		\end{enumerate}
	\end{definition}

	\begin{theorem}[Bland's Rule]
		If there are multiple entering variables, we choose the one with the
		smallest index. The simplex algorithm will always terminate if we use Bland's
		Rule. Note: this theorem was named Bland's rule because it's bland af.
	\end{theorem}

	\begin{definition*}
		\textbf{Degeneracy} \item
		\begin{definition}[Degenerate Iteration]
			\label{degeneracy:i} A degenerate iteration is an iteration where the BFS does
			not change after the iteration.
		\end{definition}

		\item
		\begin{definition}[Degenerate Solution]
			\label{degeneracy:ii} A degenerate solution is one where $x_{i}$ is zero for
			some $i \in B$.
		\end{definition}
	\end{definition*}

	\section{Geometry}
	\begin{definition}[Polyhedron]
		A polyhedron is a set of points in $\mathbb{R}^{n}$ that can be described by
		a finite number of linear inequalities (i.e. an intersection of halfspaces).
		So every polyhedron can take the form of $P = \{Ax \leq b\}$
	\end{definition}

	\begin{remark}
		The feasible region of an LP is a polyhedron.
	\end{remark}

	\chapter{Duality}

	\section{Obtaining the Dual}
	\begin{definition}[Dual of an LP]
		\label{dual} Let $P$ be an LP with objective function $c^{T}x$ and a constraint
		matrix $A$ with right-hand side $b$. Then we can use the table below to find
		the dual of $P$. The dual will have objective function $b^{T} y$, constraint
		matrix $A^{T}$, and right-hand side $c$.
	\end{definition}

	\begin{note}
		\begin{tabular}{cc}
			\begin{minipage}{.5\linewidth}\begin{tabular}{|c|c|c|}\hline \textbf{max} & & \textbf{min} \\ \hline $\leq$ constraint & $\iff$ & $\geq 0$ variable \\ \hline $\geq$ constraint & $\iff$ & $\leq 0$ variable \\ \hline $=$ constraint & $\iff$ & free variable \\ \hline \hline free variable & $\iff$ & $=$ constraint \\ \hline $\geq 0$ variable & $\iff$ & $\geq$ constraint \\ \hline $\leq 0$ variable & $\iff$ & $\leq$ constraint \\ \hline\end{tabular}\end{minipage} & \begin{minipage}{.5\linewidth}Note that for a max LP, when we go from a constraint to a variable, the sign flips. When we go from a variable to a constraint, the sign does not flip. This can be a helpful way to memorize this.\end{minipage} %
		\end{tabular}
	\end{note}

	\begin{remark}
		The dual of the dual is the primal.
	\end{remark}

	\begin{eg}
		[Finding the Dual]\label{eg} Find the dual of: \\
		\begin{equation*}
			\begin{aligned}
				\min              & \m{1 & 2 & 3}x                                                   \\
				\text{subject to} &                                                                  \\
				                  & \m{3 & 1 & 4 \\ 1 & 5 & 9 \\ 2 & 6 & 5}x \leq \m{10 \\ 24 \\ 16} \\
				                  & x_{1} \geq 0, x_{2} \leq 0, x_{3} \text{ free}
			\end{aligned}
		\end{equation*}
		\begin{answer}
			Since we are starting with a min LP, we will look at the right side of the
			table. We can see that we have $\leq$ constraints, so we must have $\leq$
			variables. Since we have $x_{1} \geq 0$, $x_{2} \leq 0$, and $x_{3}$ free,
			the constraints in the dual should be $\leq, \geq, =$ in order. So we have:
			\begin{equation*}
				\begin{aligned}
					\max              & \m{10 & 24 & 16}y                                                                                          \\
					\text{subject to} &                                                                                                            \\
					                  & \m{3 & 1 & 2 \\ 1 & 5 & 6 \\ 4 & 9 & 5}y \quad \begin{matrix}\leq \\ \geq \\ =\end{matrix} \m{1 \\ 2 \\ 3} \\
					                  & y \leq \mathbb{0}
				\end{aligned}
			\end{equation*}
		\end{answer}
	\end{eg}

	\section{Results of Duality}
	\begin{theorem}[Weak Duality Theorem]
		Let (P) be $\{\max c^{T}x : Ax = b, x\geq \mathbb{0}\}$ and (D) be its dual $\{
		\min b^{T} y : A^{T} y \geq c\}$. If $x$ and $y$ are feasible solutions to (P)
		and (D) respectively, then $c^{T} x \leq b^{T} y$. Moreover, if $c^{T} x = b^{T}
		y$, then $x$ and $y$ are optimal for (P) and (D) respectively.
	\end{theorem}
	\begin{theorem}[Strong Duality Theorem]
		Let (P) and (D) be as in the Weak Duality Theorem. If (P) has an optimal
		solution, then so does (D). Moreover, (P) and (D) will have the same optimal
		value
	\end{theorem}
	\begin{corollary}[Simple Results of Duality]
		By the duality theorems above, we have:
		\begin{enumerate}
			\item (P) is infeasible $\implies$ (D) is infeasible or unbounded.

			\item (P) is unbounded $\implies$ (D) is infeasible.

			\item (P) has an optimal solution $\implies$ (D) has an optimal solution.
		\end{enumerate}
		Reminder that the dual of the dual is the primal. For example, if we have (D)
		is infeasible, then (P) is infeasible or unbounded.
	\end{corollary}

	\section{Proving Optimality with Duality}
	Recall from \hyperref[optimality:ii]{Optimality Certficate by Weak Duality},
	that a feasible solution to the dual, $y$ can be used as a certificate of optimality
	for the primal. We can show this as follows:
	\begin{align*}
		         & Ax = b                                    \\
		\implies & \underbrace{y^TA}_{\geq^? c^T}x = y^{T} b \\
	\end{align*}
	Then we have $c^{T}x \leq^{?} y^{T} b$. Therefore, $y^{T} b$ is an upper bound
	(or lower-bound for min, hence the ?) on the optimal value of the primal. If we
	have a feasible solution $\bar x$ with value $y^{T} b$, then we know it is
	optimal.

	\section{Complementary Slackness (CS) Conditions}
	\begin{definition}[Complementary Slackness Conditions]
		Let (P) and (D) be a primal and dual pair with feasible solutions $x$ and $y$.
		Then the CS Conditions are:
		\begin{enumerate}
			\item $x_{i} = 0$ or the corresponding constraint in (D) is tight.

			\item $y_{i} = 0$ or the corresponding constraint in (P) is tight.
		\end{enumerate}
		As we saw in \hyperref[optimality:iii]{Optimality Certficate by CS
		Conditions}, if $x$ and $y$ are feasible solutions to (P) and (D)
		respectively, then $x$ and $y$ are optimal if and only if they satisfy all the
		CS conditions.
	\end{definition}

	\section{Trust Me Bro Theorem}
	This theorem did not have a name in the course notes, so I made a name for it.
	It's pretty important so it should really have a name.

	\newpage

	\begin{theorem}[Trust Me Bro Theorem]
		Let $\bar x$ be a feasible solution to $\{\max c^{T} x : Ax \leq b\}$. Then
		$\bar x$ is optimal if and only if $c$ is in the cone of tight constraints
		for $\bar x$. The generalization of this is the \hyperref[kkt]{KKT Theorem}
	\end{theorem}
	\begin{proof}
		Trust me bro.
	\end{proof}
	\begin{corollary}[Trust Me Bro Theorem Extended]
		If the LP takes the form of $\{\max c^{T} x : Ax \text{ ? }b\}$, then
		$\bar x$ is an optimal solution if and only if
		$c = \lambda_{1} g_{1} + \dots + \lambda_{k} g_{k}$, where $g_{i}$ are all the
		tight constraints for $\bar x$ and
		\begin{enumerate}
			\item $\lambda_{i} \geq 0$ for all $i = 1 \dots k$ if $Ax \leq b$.

			\item $\lambda_{i} \leq 0$ for all $i = 1 \dots k$ if $Ax \geq b$.

			\item $\lambda_{i}$ free for all $i = 1 \dots k$ if $Ax = b$.
		\end{enumerate}
	\end{corollary}
	\begin{proof}
		1. follows from the Trust Me Bro Theorem. \\[0.3in] 2. Suppose the LP is
		$\{\max c^{T} x : Ax \geq b\}$ with a feasible solution $\bar x$. Then the dual
		is \\ $\{\min b^{T} y : A^{T} y = c, y \leq \mathbb{0}\}$.\\[0.15in] ($\implies$)
		Suppose $\bar x$ is optimal. Without loss of generality, we can assume the first
		$n$ constraints are tight. Since $\bar x$ is optimal, then by CS Conditions,
		we know $y_{n+1}= \dots = y_{k} = 0$ since their corresponding constraint is
		not tight. Then from the dual, we have
		$c = A^{T}_{1} y_{1} + \dots + A^{T}_{n} y_{n}$, and $y \leq \mathbb{0}$.
		Here $A^{T}_{i}$ are the tight constraints and $y_{i}$ is $\lambda_{i}$. That
		concludes the proof.\\[0.15in] ($\impliedby$) Proof is trivial and thus omitted.
		But, trust me bro, it's true. \\[0.3in] 3. Follows a similar proof to 2.
	\end{proof}

	\chapter{Integer Programming}
	\section{Shortest Path algorithm}
	\subsection{The Problem}
	\begin{definition}[Shortest Path Problem]
		Let $G = (V, E)$ be a directed graph with $n$ vertices and $m$ edges. Let
		$c : E \rightarrow \mathbb{R}$ be a cost function. The shortest path problem
		is to find a path from $s \in V$ to $t \in V$ with the minimum cost. Without
		loss of generality, we can assume the objective function $c^{T} x$ (i.e. it is
		linear).

		\begin{corollary}[Result of Menger's Theorem]
			If $P$ is an s,t path and $\delta(S)$ is any s,t cut, then $P$ must have
			at least one edge in $\delta(S)$. This is a result of a variation of Menger's
			Theorem not shown here.
		\end{corollary}
	\end{definition}
	We can formulate the problem as an integer program as follows:
	\begin{align*}
		             & \min c^{T} x                                                       \\
		\text{s.t. } & \sum (x_{e} : e \in \delta) \geq 1 \text{ for all s,t cuts }\delta \\
		             & x_{e} \in \mathbb{N}_{0}, \text{ for all }e \in E                  \\
	\end{align*}

	\subsection{The Algorithm}
	\begin{algorithm}
		\floatname{algorithm}{Algorithm} \algrenewcommand\algorithmicrequire{\textbf{Input: }}
		\algrenewcommand\algorithmicensure{\textbf{Output: }}
		\caption{Shortest Path Algorithm}
		\label{alg:path}
		\begin{algorithmic}
			[1] \Require $G = (V, E)$, the costs: $c \in \mathbb{R}^{m}$, and the
			source $s$ and sink $t$. \Ensure an s,t path with minimum cost and the vector
			of width assignments $y$. \State Set
			$y := \mathbb{0}, U := \{s\}, T := \{\}$ \While{$t \notin U$} \State Calculate
			slack$(uv)$ for all $uv \in \delta(U)$ \State Find the edge $uv$ with
			minimum slack \\ \Comment{If there are multiple edges with the same slack, choose one arbitrarily}
			\State Set $y_{U} := \text{slack}(uv)$ \State Set $U := U \cup \{v\}$, and
			$T := T \cup \{uv\}$ \EndWhile \Comment{At the end of this loop, we have a shortest s,t path in $T$}
			\State \textbf{output} an s,t path in $T$, and the vector of width assignments
			$y$.
		\end{algorithmic}
	\end{algorithm}

	\newpage

	\section{Solving Integer Programs}
	\subsection{Properties of Integer Programs}
	\begin{definition}[Convex Hull]
		Let $S \subseteq \mathbb{R}^{n}$ be a set of points. The convex hull of $S$
		is the smallest convex set containing $S$.
		\begin{corollary}
			Each set $S$ has a unique convex hull.
		\end{corollary}
	\end{definition}

	\begin{theorem}[Meyer's Theorem]
		Let $P = \{Ax \leq b\}$, where $A$ and $b$ are rational. Then the convex hull
		of integer points in $P$ is polyhedron.
	\end{theorem}

	\begin{theorem}[IP and LP Results]
		Let (IP) be $\{\max c^{T}x : Ax \leq b, x \in \mathbb{Z}\}$, where $A, b$
		are rational. Let the convex hull of feasible solutions be the polyhedron $P
		=\{x : A' x \leq b'\}$. Also let (LP) be $\{\max c^{T} x : x \in P\}$. Then we
		have the following results:
		\begin{enumerate}
			\item (IP) is infeasible $\iff$ (LP) is infeasible.

			\item (IP) is unbounded $\iff$ (LP) is unbounded.

			\item an optimal solution to (IP) is an optimal solution to (LP).

			\item an extreme optimal solution of (LP) is an optimal solution of (IP).
		\end{enumerate}
		\begin{proof}
			No.
		\end{proof}
	\end{theorem}

	\subsection{Cutting Planes}

	\begin{definition}[Cutting Planes]
		We denote $\floor{ax = b}$ as $\floor{a}x = \floor{b}$. Let $\bar x$ be the
		optimal solution for the relaxed IP. \\ For any constraint $ax = b$, if
		$\bar x$ does not satisfy $\floor{ax \leq b}$, then $\floor{ax \leq b}$ is a
		cutting plane. It can be written as $\floor{a}x + s = \floor{b}$, where $s$ is
		a slack variable, and added to the IP.
	\end{definition}

	\subsection{Solving IPs with Relaxations}
	We can use relaxations and cutting planes to solve integer programs. The idea
	is to relax the integer program by removing the integrality constraint. Then solving
	the new linear program. If the optimal solution is an integer, then we stop.
	Otherwise, we add a cutting plane and solve again. We repeat until we find an
	integer solution. This is a really bad way of solving IPs, but oh well.

	\begin{algorithm}
		\floatname{algorithm}{Algorithm} \algrenewcommand\algorithmicrequire{\textbf{Input: } An IP: }
		\algrenewcommand\algorithmicensure{\textbf{Output: } An optimal solution to the IP: }
		\caption{Cutting Planes Algorithm}
		\label{alg:cut}
		\begin{algorithmic}
			[1] \Require $\{\max c^{T}x : Ax = b, x \in \mathbb{Z}\}$ \Ensure $\bar x$
			\State Solve the LP relaxation of the IP. Let $\bar x$ be the optimal
			solution. \While{$\bar x$ is not integer} \State Find a cutting plane $s$
			for the IP. \State Add $s$ to the IP. \State Solve the LP relaxation of
			the IP. Let $\bar x$ be the optimal solution. \EndWhile \State \textbf{return}
			$\bar x$
		\end{algorithmic}
	\end{algorithm}

	\chapter{Nonlinear Programming}

    \begin{definition}[Epigraphs]
        Let $f : \mathbb{R}^{n} \rightarrow \mathbb{R}$ be any function. The
        epigraph of $f$ is the set of all points $(x, y)$ such that $y \geq f(x)$.
        Written as $\text{epi}(f) = \Bigg\{\m{x \\ y} : y \geq f(x)\Bigg\}$. 
        
        $\trianglerighteq$ Note that $\text{epi}(f) \subseteq \mathbb{R}^{n+1}$.
        \begin{corollary}
            $f$ is convex $\iff$ $\text{epi}(f)$ is a convex set.
        \end{corollary}
    \end{definition}

	\section{Gradients and Subgradients}
	\begin{definition}[Subgradient]
		\label{subgradient} Let $f : \mathbb{R}^{n} \rightarrow \mathbb{R}$ be a convex
		function. We say $s$ is a subgradient of $f$ at $\bar x$ if $h(x) \leq f(x)$
		for all $x \in \mathbb{R}^{n}$, where $h(x) := f(\bar x) + s^{T} (x - \bar x)$
	\end{definition}
	\begin{definition}[Gradient]
		Let $f : \mathbb{R}^{n} \rightarrow \mathbb{R}$ be a convex function differentiable
		with respect to $\bar x_{i}$ for all $i = 1, \dots, n$. We say
		$\nabla f(\bar x) =
		\begin{bmatrix}
			\partial_{x_1}f, \dots, \partial_{x_n}f
		\end{bmatrix}$ is a gradient of $f$ at $\bar x$.
	\end{definition}

	\begin{proposition}[Properties of Gradients and Subgradients]
		\label{prop} . \\
		\begin{itemize}
			\item A subgradient at $\bar x$ is unique if and only if $f$ is differentiable
				at $\bar x$.

			\item A subgradient will always exist for a convex function with only bounded
				operators.

			\item Any gradient is also a subgradient, but the converse is not true.
		\end{itemize}
	\end{proposition}

	\section{NLP Relaxations}
	\begin{definition}[Supporting Halfspace]
		\label{supporting} Let $C = \{x : f(x) \leq 0\}$ and $\bar x = 0$. Then the
		supporting halfspace of $C$ is $F = \{x: h(x) \leq 0 \}$, where $h(x)$ is the
		same from \hyperref[subgradient]{Subgradient}.
	\end{definition}
	\begin{proposition}[Supporting Halfspace of a Convex Set]
		Let $C, F$ be as defined above. $C \subseteq F$.
	\end{proposition}
	\begin{proof}
		Let $x \in C$. Then $f(x) \leq 0$. Then $h(x) \leq 0$ since $h(x) \leq f(x)$
		for all $x \in \mathbb{R}^{n}$. Therefore, $x \in F$.
	\end{proof}

	\begin{corollary}[NLP Relaxation]
		\label{nlp} Let (NLP) be an NLP in the form \\$\{\min c^{T}x : g_{i}(x) \leq
		0, i = 1, \dots, m\}$. If $g_{i}(x)$ has a subgradient $s$ at $\bar x$, then
		the NLP obtained by replacing $g_{i}(x) \leq 0$ with
		$g_{i}(\bar x) + s^{T}(x - \bar x)\leq 0$ is a relaxation of (NLP).
	\end{corollary}

	\newpage

	\section{KKT Conditions}
	The KKT approach is a generalization of Lagrange Multipliers, which only allow
	for equality constraints. KKT allows for both equality and inequality constraints.
	\begin{note}
		We assume that the objective function and all constraints are differentiable
		at the optimal point.
	\end{note}
	\begin{definition*}
		\textbf{KKT Conditions} \\ \item Note that the definitions that follow are
		all equivalent, although some only apply to NLPs without equality constraints.
		\item
		\begin{definition}[KKT Conditions for a General NLP]
			\label{kkt} Given a general NLP in the form of: \\
			$\{\min f(x) : g_{i}(x) \leq 0, h_{j}(x) = 0, i = 1, \dots, m, j = 1, \dots
			, n\}$, The KKT conditions are split into 4 parts:
			\begin{enumerate}
				\item Stationarity:
					$0 \in \partial f(x) + \sum_{i=1}^{m} \lambda_{i} \partial g_{i}(x) + \sum
					_{i=1}^{n} \mu_{i} \partial h_{i}(x)$

				\item Primal Feasibility: $g_{i}(x) \leq 0$ and $h_{j}(x) = 0$ for all
					$i,j$

				\item Dual Feasibility: $\lambda_{i} \geq 0, i = 1, \dots, m$.

				\item Complementary Slackness:
					$\lambda_{i} g_{i}(x) = 0, i = 1, \dots, m$
			\end{enumerate}

			Note that $\lambda_{i}$ and $\mu_{i}$ are the Lagrange Multipliers for the
			$i$th constraint.
		\end{definition}

		%   Note that the Dual Feasibility and Stationarity conditions can be combined into one condition if a Slater Point exists: $-\nabla f(x) \in \text{cone}\left\{ g_i(x), h_j(x) : i \in T, j = 1, \dots, n \right\}$, where T is the index set of tight constraints at $x$.

		\item
		\begin{definition}[KKT Conditions for a General NLP II]
			\label{kkt2} We can rewrite the KKT conditions to make them a bit more palatable.
			Let $\bar x$ be a solution to the NLP. Also assume that the NLP has a Slater
			Point. Then the ``only" KKT condition is:
			\begin{center}
				$-\nabla f(x) \in \text{cone}\left\{ \nabla g_{i}(x), \nabla h_{j}(x) : i
				\in T, j = 1, \dots, n \right\}$,
			\end{center}
			where T is the index set of tight constraints at $\bar x$.
		\end{definition}

		\item
		\begin{definition}[KKT Conditions for an NLP without Equality Constraints]
			\label{kkt3} Given an NLP in the form of: \\
			$\{\min f(x) : g_{i}(x) \leq 0, i = 1, \dots, m\}$, The KKT conditions are
			split into 4 parts:
			\begin{enumerate}
				\item Stationarity:
					$0 \in \partial f(x) + \sum_{i=1}^{m} \lambda_{i} \partial g_{i}(x)$

				\item Primal Feasibility: $g_{i}(x) \leq 0$ for all $i$

				\item Dual Feasibility: $\lambda_{i} \geq 0, i = 1, \dots, m$.

				\item Complementary Slackness:
					$\lambda_{i} g_{i}(x) = 0, i = 1, \dots, m$
			\end{enumerate}
		\end{definition}

		\item
		\begin{definition}[KKT Conditions for an NLP without Equality Constraints II]
			\label{kkt4} Let $\bar x$ be a solution to the NLP. Also assume that the
			NLP has a Slater Point. Then the ``only" KKT condition is:
			\begin{center}
				$-\nabla f(x) \in \text{cone}\left\{ \nabla g_{i}(x): i \in T \right\}$,
			\end{center}
			where T is the index set of tight constraints at $\bar x$.
		\end{definition}
	\end{definition*}

	\begin{theorem}[KKT Theorem]
		Let $x^{*}$ be a solution to an NLP. Then $x^{*}$ is optimal if and only if
		the KKT conditions hold.
	\end{theorem}
\end{document}